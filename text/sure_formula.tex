\section{Stein's unbiased risk estimate}
\label{sec:sure_form}

Both shrinkage function (\ref{equation:trunc.shrink}) and
(\ref{equation:msst.est}) define classes of estimators, indexed by
tuning parameters.  Ideally, we would like to choose these tuning
parameters by minimizing the risk (\ref{equation:mse}). However,
because the mean $\Theta$ is unknown, minimization of
(\ref{equation:mse}) with respect to the tuning parameters is not
possible. One approach for selecting an estimator from one of these
classes is to minimize a risk estimate that does not depend on the
unknown parameter. One such estimate is Stein's unbiased risk
estimate:
\begin{theorem}[\cite{stein1981estimation}]
  \label{theorem:stein.theorem}
  Under the model (\ref{equation:normal.model}), suppose $t: \mathbb{R}^{p_1\times\cdots\times p_K} \rightarrow \mathbb{R}^{p_1\times\cdots\times p_K}$ is an almost differentiable function for which
  \begin{align}
    \label{equation:int.condition}
    E_{\Theta}\left[\sum_{\mathbf{i}} \left|\frac{d}{d\mathcal{X}_{[\mathbf{i}]}}t_{\mathbf{i}}(\mathcal{X}_{[\mathbf{i}]})\right|\right] < \infty.
  \end{align}
  Then
  \begin{align*}
    \mse(t(\mathcal{X})) = E_{\Theta}[||\Theta - t(\mathcal{X})||^2] = E_{\Theta}\left[ ||t(\mathcal{X}) - \mathcal{X}||^2 + 2\tau^2\diverge(t(\mathcal{X})) - p\tau^2\right],
  \end{align*}
  where $\diverge(\cdot)$ is the divergence of $t(\cdot)$. We denote
  Stein's unbiased risk estimate (SURE) as
  \begin{align}
    \label{equation:sure.form}
    \sure(t) = ||t(\mathcal{X}) - \mathcal{X}||^2 + 2\tau^2\diverge(t(\mathcal{X})) - p\tau^2.
  \end{align}
\end{theorem}

``Almost differentiable'' basically means differentiable everywhere
except on a set of Lebesgue measure zero \citep[Definition
1]{stein1981estimation}. Because the SURE (\ref{equation:sure.form})
does not depend on the parameter values $\Theta$, we can minimize the
SURE and use this minimization as a proxy for minimizing the risk. In
many cases, adaptive estimators obtained by minimizing SURE over a
class of estimators yields improved risk performance, as was observed
by \cite{candes2013unbiased} in the matrix case.


The difficult part of (\ref{equation:sure.form}) is calculating the
divergence. We will spend the next two subsections performing this
task. First, we will calculate the differentials for the elements of
the altered HOSVD (\ref{equation:hosvd.rewrite}) in Subsection
\ref{subsection:diff}. Then we will use these differentials to derive
the divergence of estimators of the form (\ref{equation:ho.spect.est})
in Subsection \ref{subsection:div}. This divergence can then be
inserted into (\ref{equation:sure.form}) to obtain the SURE.





\subsection{Differentials of the HOSVD}
\label{subsection:diff}
In this subsection, we calculate the differentials for the elements in
the altered HOSVD (\ref{equation:hosvd.rewrite}). In what follows, we
will assume that $\mathcal{X}$ has full multilinear rank. Given that
$p_k \leq p/p_k$ for all $k = 1,\ldots,K$, where $p =
\prod_{k=1}^Kp_k$, this rank condition is fulfilled almost surely for
data $\mathcal{X}$ that have a p.d.f. that is absolutely continuous
with respect to Lebesgue measure on $\mathbb{R}^{p_1\times\cdots\times
  p_K}$ \cite[Proposition 7.2]{de2008tensor}.

\begin{theorem}
  \label{theorem:diff}
  The differentials of $D_k$, $U_k$, and $\mathcal{V}$ from
  (\ref{equation:hosvd.rewrite}) are given in equations
  (\ref{equation:d.D}), (\ref{equation:d.U}), and
  (\ref{equation:d.V}), respectively.
\end{theorem}


An outline of the derivation is as follows: Because each $U_k$ and
$D_k$ from the HOSVD is from the SVD of $\mathcal{X}_{(k)} =
U_kD_kV_k^T$, the calculation begins by recognizing that the
differentials of the $U_k$'s and the $D_k$'s are the same as in the
matrix case. The differentials can then be re-written as functions of
the terms in the HOSVD. To obtain the differential of $\mathcal{V}$,
we write $\mathcal{X} =
(U_1,\ldots,U_K)\cdot(D_1,\ldots,D_K)\cdot\mathcal{V}$ and apply the
chain rule to each $U_k$, each $D_k$, then to $\mathcal{V}$. We then
solve for the differential of $\mathcal{V}$, which may be written in
terms of the differentials of the $U_k$'s and the $D_k$'s.

\begin{proof}[Proof of Theorem \ref{theorem:diff}]
  Denote the differential of a function $g$ at $\mathcal{X}$ with
  increment $\Delta$ as $dg[\Delta]$. Since $U_k$ and $D_k$ are the
  left singular vectors and the singular values, respectively, of
  $\mathcal{X}_{(k)}$ for each $k = 1,\ldots,K$, the differentials,
  $dU_k[\Delta]$ and $dD_k[\Delta]$, are the same as in
  \cite{candes2013unbiased} and have a closed form solution, given by
  \begin{align}
    \label{equation:d.D}
    d\sigma_i^k[\Delta] = (U_k^T\Delta_{(k)}U_{-k}\mathcal{S}_{(k)}D_k^{-1})_{[i,i]} \text{ for } i =1,\ldots,p_k \text{ and } k=1,\ldots,K,
  \end{align}
  where
  \begin{align*}
    U_{-k} = U_K\otimes \cdots \otimes U_{k+1} \otimes U_{k-1}\otimes\cdots\otimes U_1.
  \end{align*}
  This follows because the SVD of $\mathcal{X}_{(k)}$ is $U_kD_kV_k^T
  = U_k\mathcal{S}_{(k)}U_{-k}^T$ which implies that $V_k =
  U_{-k}\mathcal{S}_{(k)}^TD_{k}^{-1}$. We plug in $V_k$ into equation
  (4.7) of \cite{candes2013unbiased} to get (\ref{equation:d.D}).

  Let $\Omega_{U_k}[\Delta] = U_k^TdU_k[\Delta]$. Then from (4.8) of
  \cite{candes2013unbiased} we have
  \begin{align}
    \begin{split}
      \label{equation:d.omega}
      &\Omega_{U_k}[\Delta]_{[i,j]} \\
      &= - 1(i \neq j)\left[\sigma_j^k(U_k^T\Delta_{(k)}U_{-k}S_{(k)}^TD_k^{-1})_{[i,j]} + \sigma_i^k(U_k^T\Delta_{(k)}U_{-k}S_{(k)}^TD_k^{-1})_{[j,i]}\right]/((\sigma_i^k)^2 - (\sigma_j^k)^2),
    \end{split}
  \end{align}
  and so
  \begin{align}
    \label{equation:d.U}
    dU_k[\Delta] = U\Omega_{U_k}[\Delta].
  \end{align}

  We now derive $d\mathcal{V}[\Delta]$. Let $U = (U_1,\ldots,U_K)$ and
  $D = (D_1,\ldots,D_K)$. Also note that $d\mathcal{X}[\Delta] =
  \Delta$. Using the chain rule, and following Chapter 8, Section 1,
  Equations (15) and (16) of \cite{magnus1999matrix} for the
  differential of matrix multiplication and the Kronecker product, we
  have
  \begin{align}
    \Delta = d\mathcal{X}[\Delta] &= d(U\cdot D \cdot\mathcal{V})[\Delta] \nonumber\\
    &=\sum_{k=1}^Kd\underline{U}_k[\Delta] \cdot D \cdot \mathcal{V} + \sum_{k=1}^K U \cdot d\underline{D}_k[\Delta] \cdot \mathcal{V} + U \cdot D \cdot d\mathcal{V}[\Delta], \label{equation:chain.rule}
  \end{align}
  where
  \begin{align}
    &d\underline{U}_k[\Delta] = (U_1,\ldots,U_{k-1},dU_k[\Delta],U_{k+1},\ldots,U_K) \text{ and} \label{equation:U.underline}\\
    &d\underline{D}_k[\Delta] = (D_1,\ldots,D_{k-1},dD_k[\Delta],D_{k+1},\ldots,D_K).
  \end{align}

  From (\ref{equation:chain.rule}), we solve for
  $d\mathcal{V}[\Delta]$ and have
  \begin{align}
    \label{equation:d.V}
    d\mathcal{V}[\Delta] = D^{-1} \cdot U^T \cdot \Delta - \sum_{k=1}^K dF_k[\Delta]\cdot \mathcal{V} - \sum_{k=1}^K dG_k[\Delta] \cdot \mathcal{V},
  \end{align}
  where
  \begin{align}
    \label{equation:d.FG}
    &dF_k[\Delta] = (I_{p_1},\ldots,I_{p_{k-1}},D_k^{-1}\Omega_{U_k}[\Delta]D_k,I_{p_{k+1}},\ldots,I_{p_K}) \text{ and}\\
    &dG_k[\Delta] = (I_{p_1},\ldots,I_{p_{k-1}},D_k^{-1}dD_k[\Delta],I_{p_{k+1}},\ldots,I_{p_K}).
  \end{align}
\end{proof}

\subsection{Divergence of higher-order spectral estimators}
\label{subsection:div}
In this section, we show that the divergence of higher-order spectral
estimators of the form (\ref{equation:ho.spect.est}) can be found in
the following theorem.
\begin{theorem}
  The divergence of estimators of the form
  (\ref{equation:ho.spect.est}) is
\begin{align}
  \sumo\left(f(D) \cdot D^{-1}\cdot \mathcal{C} + \sum_{k=1}^KH_k \cdot \mathcal{S}^2\right), \label{equation:divergence.final}
\end{align}
where $\sumo(\mathcal{A})$ is the sum of all elements in the tensor
$\mathcal{A}$, $\mathcal{S}^2 \in \mathbb{R}^{p_1\times\cdots\times
  p_K}$ such that $(\mathcal{S}^2)_{[\mathbf{i}]} =
(\mathcal{S}_{[\mathbf{i}]})^2$,
\begin{align}
  \label{equation:H.k}
  H_k = (f^1(D_1)D_1^{-1},\ldots,f^{k-1}(D_{k-1})D_{k-1}^{-1},D_{k}^{-1}df^k(D_k)D_k^{-1},f^{k+1}(D_{k+1}),\ldots,f^K(D_K)),
\end{align}
and $\mathcal{C} \in \mathbb{R}^{p_1\times\cdots\times p_K}$ such that
\begin{align}
  \label{equation:C.array}
  \mathcal{C}_{[{\bf i}]} =  1 + \sum_{k=1}^K \sum_{j = 1, j\neq i_k}^{p_k} \frac{\mathcal{S}_{[i_1,\ldots,i_{k-1},j,i_{k+1},\ldots,i_K]}^2}{(\sigma_{i_k}^k)^2 - (\sigma_{j}^k)^2} - \mathcal{S}_{\bf [i]}^2\sum_{k=1}^K\left( \frac{1}{(\sigma_{i_k}^k)^2} + \sum_{m=1,m\neq i_k}^{p_k}\frac{1}{(\sigma_m^k)^2 - (\sigma_{i_k}^k)^2}  \right).
\end{align}
\end{theorem}

\begin{proof}
  Let
  \begin{align*}
    \Delta^{i_1,\ldots,i_K} = \Delta^{\bf i} = U_{1[:,i_1]} \circ \cdots \circ U_{K[:,i_K]},
  \end{align*}
  where $\circ$ is the outer product and $U_{k[:,i_k]}$ is the $i_k$th
  column of $U_k$. Note that
  \begin{align*}
    (U_1^T,\ldots,U_K^T)\cdot\Delta^{\bf i} = E^{\bf i},
  \end{align*}
  where $E^{\bf i}$ is the $p_1\times\cdots\times p_K$ array with a
  one in position $(i_1,\ldots,i_K)$ and zeros everywhere
  else. Similar to the arguments of \cite{candes2013unbiased}, also
  note that $\Delta^{\bf i}$ forms an orthonormal basis for
  $\mathbb{R}^{p_1\times\cdots\times p_K}$, and so
  \begin{align}
    \diverge(t(\mathcal{X})) &= \sum_{\bf i} \langle\Delta^{\bf i}, df[\Delta^{\bf i}]\rangle \nonumber\\
    &= \sum_{\bf i} \langle(U_1^T,\ldots,U_K^T)\cdot\Delta^{\bf i}, (U_1^T,\ldots,U_K^T)\cdot df[\Delta^{\bf i}]\rangle\nonumber\\
    &= \sum_{\bf i} \langle E^{\bf i}, (U_1^T,\ldots,U_K^T)\cdot df[\Delta^{\bf i}]\rangle,\nonumber\\
    \label{equation:divergence.index}
    &= \sum_{\bf i} ((U_1^T,\ldots,U_K^T)\cdot df[\Delta^{\bf i}])_{[\mathbf{i}]},
  \end{align}
  where $\langle , \rangle$ is the usual Euclidean inner product. From
  the chain rule, we have:
  \begin{align*}
    df[\Delta^{\bf i}] = \sum_{k=1}^Kd\underline{U}_k[\Delta^{\bf i}] \cdot f(D) \cdot \mathcal{V} + \sum_{k=1}^K U \cdot df(\tilde{D})_k[\Delta^{\bf i}] \cdot \mathcal{V} + U \cdot f(D) \cdot d\mathcal{V}[\Delta^{\bf i}],
  \end{align*}
  where
  \begin{align*}
    &f(D) = (f^1(D_1),\ldots,f^K(D_K)) \text{ and}\\
    &df(\tilde{D})_k[\Delta^{\bf i}] = (f^1(D_1),\ldots,f^{k-1}(D_{k-1}),d(f^k \circ D_k)[\Delta^{\bf i}],f^{k+1}(D_{k+1}),\ldots,f^K(D_K)),
  \end{align*}
  where ``$\circ$'' now means composition. Hence,
  \begin{align}
    \label{equation:u.times.df}
    U^T \cdot df[\Delta^{\bf i}] = \sum_{k=1}^Kd\tilde{U}_k[\Delta^{\bf i}] \cdot f(D) \cdot \mathcal{V} + \sum_{k=1}^K df(\tilde{D})_k[\Delta^{\bf i}] \cdot \mathcal{V} + f(D) \cdot d\mathcal{V}[\Delta^{\bf i}],
  \end{align}
  where
  \begin{align}
    d\tilde{U}_k[\Delta^{\bf i}] = (I_{p_1},\ldots,I_{p_{k-1}},\Omega_{U_k}[\Delta^{\bf i}],I_{p_{k+1}},\ldots,I_{p_K}). \label{equation:U.tilde}
  \end{align}

  The outline of the derivation of the divergence is as follows. The
  ultimate goal is to obtain the $(i_1,\ldots,i_K)$th element of $U^T
  \cdot df[\Delta^{\bf i}]$ in (\ref{equation:u.times.df}) and plug
  that into (\ref{equation:divergence.index}). We will first calculate
  all of the differentials that are in (\ref{equation:u.times.df}),
  then we will determine the $(i_1,\ldots,i_K)$th element of $U^T
  \cdot df[\Delta^{\bf i}]$. Then we will simplify
  (\ref{equation:divergence.index}). These latter two steps may be
  found in Appendix \ref{sec:simp.div}.

  We begin with the differentials. From (\ref{equation:d.D}), we have
  \begin{align}
    d\sigma_j^k[\Delta^{\bf i}] &= (U_k^T \Delta_{(k)}^{\bf i} U_{-k} S_{(k)}^T D_k^{-1})_{[j,j]} \nonumber\\
    &= (E^{\bf i}_{(k)} S_{(k)}^T D_k^{-1})_{[j,j]}\nonumber\\
    \label{equation:d.sigma.i}
    &= 1(j = i_k)S_{[i_1,\ldots,i_{k-1},j,i_{k+1},\ldots,i_K]}/\sigma_j^k.
  \end{align}
  This is since $E_{(k)}^{\bf i}S_{(k)}^T \in \mathbb{R}^{p_k\times p_k}$ such that
  \begin{align}
    \left(E_{(k)}^{\bf i}S_{(k)}^T\right)_{[\ell,j]} = \label{equation:Ek.Sk}
    \begin{cases}
      0 &\text{if } \ell \neq i_k\\
      S_{[i_1,\ldots,i_{k-1},j,i_{k+1},\ldots,i_K]} &\text{if } \ell = i_k.
    \end{cases}
  \end{align}
  Similarly, from (\ref{equation:d.omega}), we have
  \begin{align}
    &\Omega_{U_k}[\Delta^{\bf i}]_{[\ell, j]} \nonumber\\
    &= -1(\ell\neq j)\left[\sigma_j^k(U_k^T\Delta_{(k)}U_{-k}S_{(k)}^TD_k^{-1})_{[\ell,j]} + \sigma_{\ell}^k(U_k^T\Delta_{(k)}U_{-k}S_{(k)}^TD_k^{-1})_{[j,\ell]}\right]/((\sigma_{\ell}^k)^2 - (\sigma_j^k)^2)\nonumber\\
    &= -1(\ell\neq j)\left[\sigma_j^k(E^{\bf i}_{(k)}S_{(k)}^TD_k^{-1})_{[\ell,j]} + \sigma_{\ell}^k(E^{\bf i}_{(k)}S_{(k)}^TD_k^{-1})_{[j,\ell]}\right]/ ((\sigma_{\ell}^k)^2 - (\sigma_j^k)^2)\nonumber\\
    \label{equation:d.omega.i}
    &= -1(\ell\neq j)\left[S_{[i_1,\ldots,i_{k-1},j,i_{k+1},\ldots,i_K]}1(\ell = i_k) + S_{[i_1,\ldots,i_{k-1},\ell,i_{k+1},\ldots,i_K]}1(j=i_k)\right]/((\sigma_{\ell}^k)^2 - (\sigma_j^k)^2).
  \end{align}
  Also, from the chain rule, we have that
  \begin{align}
    d(f^k_j \circ \sigma^k_j)[\Delta^{\bf i}] &= \left(\frac{d}{d\sigma_j^k}f_j^k(\sigma_j^k)\right)d\sigma_j^k[\Delta^{\bf i}]\nonumber\\
    \label{equation:d.comp}
    &= \delta_{j,i_k}\left(\frac{d}{d\sigma_j^k}f_j^k(\sigma_j^k)\right)S_{[i_1,\ldots,i_{k-1},j,i_{k+1},\ldots,i_K]}/\sigma_j^k.
  \end{align}

  We have just completed all of the calculus necessary to obtain the
  divergence, and the remainder of the calculation is
  simplification. That is, we can use equations (\ref{equation:d.V}),
  (\ref{equation:divergence.index}), (\ref{equation:u.times.df}),
  (\ref{equation:d.sigma.i}), (\ref{equation:d.omega.i}), and
  (\ref{equation:d.comp}) to calculate a closed-form expression for
  the divergence. This simplification is relegated to Appendix
  \ref{sec:simp.div}.
\end{proof}

% Note that this form of the divergence is beneficial for numerical stability reasons because we can calculate first
% $f^k(D_k)D_k^{-1}$ for $k = 1,\ldots,K$, then Tucker product this list
% of matrices with an array that is a sum of elements that should not get
% too small (since they only contain ratios of $\mathcal{S}^2$ and the
% mode-specific singular values).


We now present the formula for the SURE for all higher-order spectral
estimators of the form (\ref{equation:ho.spect.est}):
\begin{theorem}[SURE for (\ref{equation:ho.spect.est})]
  Under the model (\ref{equation:normal.model}), suppose $t(\cdot)$ in
  (\ref{equation:ho.spect.est}) is almost differentiable and for which
  (\ref{equation:int.condition}) holds. Then
  \begin{align}
    \sure(t) =  ||t(\mathcal{X}) - \mathcal{X}||^2 + 2\tau^2\sumo\left(f(D) \cdot D^{-1}\cdot \mathcal{C} + \sum_{k=1}^KH_k \cdot \mathcal{S}^2\right) - p\tau^2. \label{equation:complete.sure}
  \end{align}
\end{theorem}

This SURE formula is applicable for all shrinkage functions of the
form (\ref{equation:ho.spect.est}) where $f^k(D_k) =
\diag(f_1^k(\sigma_1^k),\ldots,f_{p_k}^k(\sigma_{p_k}^k))$. For such
shrinkage functions, the shrinkage being applied to each singular
value is a function only of that singular value. However, it is
possible to construct estimators which use all of the mode $k$
singular values to shrink each mode $k$ singular value, e.g.\ if we
were to use a shrinkage function analogous to those of
(\ref{equation:stein}) or (\ref{equation:improved.em}). For such
estimators, we prove in Appendix \ref{sec:gen.spec.func} that the form
of the divergence is very similar as in
(\ref{equation:divergence.final}). The only difference is that one
replaces $\frac{d}{d\sigma_{i_k}^k}f_{i_k}^k(\sigma_{i_k}^k)$ with
$\frac{d}{d\sigma_{i_k}^k}f_{i_k}^k(\sigma_{1}^k,\ldots,\sigma_{p_k}^k)$. That
is, for such shrinkage functions, $df^k(D_k)$ is a diagonal matrix
containing only the diagonal of the Jacobian matrix of the
transformation $\diag(D_k) \mapsto \diag(f(D_k))$.
