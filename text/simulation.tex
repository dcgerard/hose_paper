\section{Simulation studies}
\label{sec:simulation}

In this section, we consider four competitors to the mode-specific
soft-thresholding estimator (\ref{equation:msst.est}) and the
truncated HOSVD (\ref{equation:trunc.shrink}). We will compare these
estimators assuming the error variance $\tau^2$ is one. The first
competitor is $\mathcal{X}$, which is the maximum likelihood estimator
and the uniformly minimum variance unbiased estimator. However, the
risk-performance of this estimator is known to be dominated by our
second competitor, the James-Stein estimator (\ref{equation:stein})
\citep{stein1981estimation}.  This estimator may be derived from an
empirical Bayes argument where $\Theta_{[\mathbf{i}]} \sim
N(0,\gamma^2)$ \citep{efron1972limiting}. As such, it should perform
well when the entries of $\Theta$ are centered about $0$. For a matrix
parameter $\Theta$, \cite{efron1972empirical} developed an empirical
Bayes estimator that performs better than the James-Stein estimator
when $\Theta$ exhibits empirical correlation along the rows. With this
in mind, our third estimator is obtained by applying the Efron-Morris
estimator (\ref{equation:efron.morris}) to the first mode
matricization of the data tensor. However, the Efron-Morris estimator
does not induce low rank estimates, and so our fourth and final
competitor is the matrix soft-thresholding estimator
(\ref{equation:soft.thresholding}) applied to the first mode
matricization of $\mathcal{X}$, and whose tuning parameter is chosen
with the SURE formula from \cite{candes2013unbiased}. This estimator
should improve on the Efron-Morris estimator when $\Theta_{(1)}$ has
approximately low rank.

We now describe the design of the simulation study. We evaluated the
risk of the mode-specific soft-thresholding, truncated HOSVD, maximum
likelihood, James-Stein, Efron-Morris, and matrix soft-thresholding
estimators under six different values of $\Theta \in \mathbb{R}^{10
  \times 10 \times 10}$, constructed as follows:
\begin{description}[noitemsep]
\item[A.] $\veco(\Theta) \sim N_p(0,I_{1000})$.
\item[B.] $\veco(\Theta) \sim N_p(0,I_{10} \otimes I_{10} \otimes F)$, where $F =
  \diag(1^2,2^2,\ldots,10^2)$.
\item[C.] $\veco(\Theta) \sim N_{1000}(0,I_{10}\otimes I_{10}\otimes \Sigma)$ where
  $\Sigma \in \mathbb{R}^{10\times 10}$ has an AR-1 $(0.7)$ covariance structure. That is,
  $\Sigma_{[i,j]} = 0.7^{|i - j|}$.
\item[D.] $\Theta_{(1)} = U_{[:,1:5]}D_{[1:5,1:5]}V_{[:,1:5]}^T$ where
  $UDV^T$ is the SVD of a $10 \times 100$ matrix that has standard
  normal entries.
\item[E.] $\veco(\Theta) \sim N_p(0,F \otimes F \otimes F)$, where $F =
  \diag(1^2,2^2,\ldots,10^2)$.
\item[F.] $\Theta$ is a rank $(5,5,5)$ tensor where all of the non-zero mode-specific
  singular values are the same along all modes.
\end{description}
% To get such a $\Theta$, we first drew $\veco(\mathcal{Y}) \sim
% N_{125}(0,I_{125})$ where $\mathcal{Y} \in \mathbb{R}^{5\times
% 5\times 5}$. We then found the ISVD of $Y =
% \ell(U_1,U_2,U_3)\cdot(D_1,D_2,D_3)\cdot\mathcal{V}$ from
% \cite{gerard2014higher}, which is different from
% (\ref{equation:hosvd.rewrite}). We then Set $\Theta =
% (\tilde{U}_1,\tilde{U}_2,\tilde{U}_{3})\cdot\mathcal{V}$, where
% $\tilde{U}_k$ was drawn uniformly from the Stiefel manifold of $10
% \times 5$ matrices.
For each scenario, we re-scaled $\Theta$ to have Frobenius norm
$\sqrt{1000}$, so that $E[||\mathcal{E}||^2] = 1000 =
||\Theta||^2$. For each $\Theta$, we simulated
$\mathcal{X}_{[\mathbf{i}]} \sim N(\Theta_{[\mathbf{i}]},1)$,
calculated the six estimators given this data tensor, and calculated
the squared error loss for each estimator. We repeated this process
500 times. Box plots of the losses for each of the six $\Theta$ values
are given in Figure \ref{fig:sim.results}.

The James-Stein estimator (\ref{equation:stein}) is expected to
perform well in Scenario \textbf{A} as it can be viewed as an
empirical Bayes procedure for the prior with which $\Theta$ was
actually generated.
% the mean was sampled under the conditions that the empirical Bayes
% estimator was derived.
Indeed, from Figure \ref{fig:sim.results} (\textbf{A}), the
James-Stein estimator does perform best, but the the mode-specific
soft-thresholding estimator performs almost as well, even though there
is no correlation along any of the modes of the mean tensor.

For scenario \textbf{B}, we expect the matrix soft-thresholding
estimator (\ref{equation:soft.thresholding}) to do well. Since the
mean tensor in this scenario has approximately low rank only along the
first mode, estimators that shrink towards the space of low
multilinear rank tensors should be over-fitting and should not perform
well. From Figure \ref{fig:sim.results} (\textbf{B}), the matrix
soft-thresholding estimator does perform best, but the mode-specific
soft-thresholding estimator does surprisingly well, and performs just
below that of the matrix soft-thresholding estimator.

For Scenario \textbf{C}, we expect the matrix soft-thresholding
estimator (\ref{equation:soft.thresholding}) and the Efron-Morris
estimator (\ref{equation:efron.morris}) to perform well. There is
temporal correlation along one of the modes of the mean tensor. We
take into account the temporal correlation of the mean by performing
soft-thresholding along this mode. However, from Figure
\ref{fig:sim.results} (\textbf{C}), we see that the mode-specific
soft-thresholding estimator performed best.

The matrix soft-thresholding estimator
(\ref{equation:soft.thresholding}) was designed to do well when the
mean matrix is of low rank. This is exactly the situation in Scenario
\textbf{D}, as a tensor with low rank along one mode may be matricized
to form a low rank matrix. However, from Figure \ref{fig:sim.results}
(\textbf{D}), for our one $\Theta$ value, the mode-specific
soft-thresholding estimator performs best.

As for Scenario \textbf{E}, we expect the mode-specific
soft-thresholding estimator (\ref{equation:msst.est}) to do well, as
the mean tensor has approximately low multilinear rank, but it is not
exactly low multilinear rank. Figure \ref{fig:sim.results}
(\textbf{E}) reveals the the mode-specific soft-thresholding estimator
does indeed perform better than the other estimators.

We expect the truncated HOSVD (\ref{equation:trunc.shrink}) to do well
in Scenario \textbf{F} because the mean tensor has low multilinear
rank, and the truncated HOSVD is correctly shrinking toward this
structure. From Figure \ref{fig:sim.results} (\textbf{F}), we see that
the truncated HOSVD does indeed perform best in terms of
loss. However, the mode-specific soft-thresholding estimator does not
perform much worse. The estimators that do not take into account the
tensor indexing perform about twice as bad as these tensor-specific
estimators.

For scenarios \textbf{C} and \textbf{D}, we emphasize here that we are
looking at the risk only at a few points in the parameter space. There
are likely points where the matrix-soft thresholding estimator
performs better than the tensor estimators. However our mode-specific
soft-thresholding estimator did not perform poorly under any of our
simulated mean tensors.


\begin{figure*}
  \begin{center}
    \includegraphics{./fig/different_scenario_sims_losses_combined2}
    \caption{Box plots of losses for the six estimators under different
      scenarios. The estimators include the mode-specific
      soft-thresholding (ST), truncated HOSVD (Tr), matrix
      soft-thresholding (MS), Efron-Morris (EM), James-Stein (JS), and
      maximum likelihood (X) estimators. In the scenarios, the mean tensor
      was simulated to have (\textbf{A}) uncorrelated elements,
      (\textbf{B}) full rank but dispersed singular values only along mode
      1, (\textbf{C}) AR-1 covariance along mode 1, (\textbf{D}) low rank
      only along mode 1, (\textbf{E}) full rank but dispersed singular
      values along all modes, and (\textbf{F}) rank $(5,5,5)$ with all the
      same non-zero singular values.}
    \label{fig:sim.results}
  \end{center}
\end{figure*}

Our procedure for the truncated HOSVD produces a multilinear rank with
the smallest SURE. It is of interest to know if this multilinear rank
provides a good estimate of the true rank of $\Theta$. We evaluated
this possibility in simulation Scenarios \textbf{D} and \textbf{F}. In
Scenario \textbf{F}, where the tensor had dimension $(10,10,10)$ and
the true multilinear rank was $(5,5,5)$, this SURE method correctly
estimated the multilinear rank in 92.6\% of trials. In Scenario
\textbf{D}, where the true multilinear rank was $(5,10,10)$, the
results of the simulation study can be found in Table
\ref{tab:rank.est}. There, we see that the rank of the first mode is
correctly estimated in 97\% of trials. The rank of the second and
third modes are correctly estimated a majority of the time.

% latex table generated in R 3.3.2 by xtable 1.8-2 package
% Mon Feb 13 10:53:01 2017
\begin{table}[ht]
\centering
\begin{tabular}{rlllllll}
  \hline
 Estimated Rank & 4 & 5 & 6 & 7 & 8 & 9 & 10 \\
  \hline
Mode 1 & .03 & .97 & 0 & 0 & 0 & 0 & 0 \\
  Mode 2 & 0 & 0 & .02 & .03 & .11 & .27 & .57 \\
  Mode 3 & 0 & 0 & 0 & .01 & .05 & .18 & .74 \\
   \hline
\end{tabular}
\caption{Proportion of times each rank is estimated based on SURE for each mode over 500 repetitions when the true multilinear rank is (5, 10, 10).}
\label{tab:rank.est}
\end{table}
